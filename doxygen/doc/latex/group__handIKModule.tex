\section{hand\+I\+K\+Module}
\label{group__handIKModule}\index{hand\+I\+K\+Module@{hand\+I\+K\+Module}}


A module that, given three 3D points and their normals, find the configuration of the hand to make thumb, index and middle reach those three points.  


A module that, given three 3D points and their normals, find the configuration of the hand to make thumb, index and middle reach those three points. 

For further information\+:

I.\+Gori, U. Pattacini, V. Tikhanoff, G. Metta Three-\/\+Finger Precision Grasp on Incomplete 3D Point Clouds. In Proceedings of I\+E\+EE International Conference on Robotics and Automation (I\+C\+RA), 2014.\hypertarget{group__handIKModule_intro_sec}{}\subsection{Description}\label{group__handIKModule_intro_sec}
This module, given triplet of 3D points, returns the configuration of the hand in terms of joints position along with the end-\/effector position and orientation, so that thumb, index and middle fingers reach those three points.\hypertarget{group__handIKModule_rpc_port}{}\subsection{Commands\+:}\label{group__handIKModule_rpc_port}
The commands sent as bottles to the module port /$<$mod\+Name$>$/rpc are described in the following\+:

{\bfseries IK} ~\newline
format\+: \mbox{[}I\+Kparam center (x y z) dim (x y z) c1 (x y z) c2 (x y z) c3 (x y z) n1 (x y z) n2 (x y z) n3 (x y z) rot (r1 r2 r3 r4 r5 r6 r7 r8 r9)\mbox{]} ~\newline
action\+: param can be 1, 2, 3 or 4. For this module it doesn\textquotesingle{}t make any difference, but it is useful for the precision-\/grasp module. c is the center of the object, dim represents the dimension of the object, found using minimum\+Bounding\+Box, c1, c2 and c3 are the positions of the points of the triplet, n1 n2 and n3 are the normals, rot is the rotation matrix betwee the object reference frame and the robot reference frame.\hypertarget{group__handIKModule_lib_sec}{}\subsection{Libraries}\label{group__handIKModule_lib_sec}

\begin{DoxyItemize}
\item Y\+A\+RP libraries.
\item I\+P\+O\+PT library
\end{DoxyItemize}\hypertarget{group__handIKModule_portsc_sec}{}\subsection{Ports Created}\label{group__handIKModule_portsc_sec}

\begin{DoxyItemize}
\item {\itshape /} $<$mod\+Name$>$/rpc remote procedure call. It always replies something.
\item {\itshape /} $<$mod\+Name$>$/$<$hand$>$/out this is the port that replies with the solution found by the module. It replies with the following format\+: \mbox{[}hand h cost c ee (x y z) or (x y z a) joints (j1 j2 j3 j4 j5 j6 j7 j8) combination (a b c)\mbox{]}. Hand is the hand for which the inverse kinematics problem has been solved. cost is the best value of the objective function. ee is the end effector position and or is its orientation. joints represent the joint angles, and combination tells you which point has to be reached by the thumb, which from the index and which from the middle finger.
\end{DoxyItemize}\hypertarget{group__handIKModule_parameters_sec}{}\subsection{Parameters}\label{group__handIKModule_parameters_sec}
The following are the options that are usually contained in the configuration file\+:

--name {\itshape name} 
\begin{DoxyItemize}
\item specify the module name, which is {\itshape hand\+I\+K\+Module} by default.
\end{DoxyItemize}

--robot {\itshape hand} 
\begin{DoxyItemize}
\item specify the hand for which the problem is being solved.
\end{DoxyItemize}\hypertarget{group__handIKModule_tested_os_sec}{}\subsection{Tested OS}\label{group__handIKModule_tested_os_sec}
Windows, Linux

\begin{DoxyAuthor}{Author}
Ilaria Gori 
\end{DoxyAuthor}
